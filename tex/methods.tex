\chapter{Implementation} % (fold)
\label{cha:implementation}
\section{Alignment} % (fold)
\label{sec:alignment}
different alignment methods.

\subsection{Dynamic Time Warping} % (fold)
\label{sub:dynamic_time_warping}
Dynamic time warping (DTW).
\begin{figure}[htbp]
	\begin{center}
		\includegraphics[width=.8\textwidth]{fig/dtw.pdf}
		\caption{Distance Matrix used in Dynamic Time Warping}
		\label{fig:dtw}
	\end{center}
\end{figure}


% subsection Dynamic Time Warping (end)
% section Alignment (end)

\section{Classification} % (fold)
\label{sec:classification}
number of classes
% section Classification (end)

\section{Training Data} % (fold)
\label{sec:training_data}
amount of data
% section Training data (end)

\section{Conversion Function Parameters} % (fold)
\label{sec:conversion_function_parameters}
The matrices in \eqref{eq:param_computed} could get really huge with a large amount of mixture models and long signal vectors. To simplify the computational complexity we can constrain the covariance matrices to be diagonal \cite{stylianou98}. $\mathbf{\Sigma}^{yx}$ will become a $m\times p$ matrix and $\mathbf{D}$ will have dimensions $n\times m$. The new transfer function will then become
\begin{equation}
	\mathcal{F}(x_t^{(k)}) = \sum_{i=1}^{m}P(C_i \vert \mathbf{x}_t)[\mu_i^{y^{(k)}}+\sigma_i^{yx^{(k)}}  (x_t^{(k)}-\mu_i^{x^{(k)}})/\sigma_i^{xx^{(k)}}]
\end{equation}
where $k=1,\dots,p$ and denotes one vector parameter.

We split the optimisation problem into $p$ independent problems which means we have to calculate the unknown vectors $\boldsymbol{\sigma}^{yx}$ and $\boldsymbol{\mu}^{y}$ for each $k$. They both will have dimensions $m\times 1$ for each $k$.

The $\mathbf{D}^{(k)}$ must also be modified to fit this new scheme.
\begin{equation}
	\label{eq:D_matrix_new}
	\mathbf{D}^{(k)} = \begin{bmatrix}
		p_1(1)(x_1^{(k)} - \mu_1^{x^{(k)}})/\sigma_1^{xx^{(k)}} & \dots & p_1(m)(x_1^{(k)} - \mu_m^{x^{(k)}})/\sigma_m^{xx^{(k)}} \\
		p_2(1)(x_2^{(k)} - \mu_1^{x^{(k)}})/\sigma_1^{xx^{(k)}} & \dots & p_2(m)(x_2^{(k)} - \mu_m^{x^{(k)}})/\sigma_m^{xx^{(k)}} \\
		\vdots & & \vdots \\
		p_n(1)(x_n^{(k)} - \mu_1^{x^{(k)}})/\sigma_1^{xx^{(k)}} & \dots & p_n(m)(x_n^{(k)} - \mu_m^{x^{(k)}})/\sigma_m^{xx^{(k)}} \\
	\end{bmatrix}
\end{equation}
where $p_n(m)=P(C_m\vert \mathbf{x}_n)$.

To compute the parameters in the transfer function the dimensions in \eqref{eq:param_computed} with the new variables $\boldsymbol{\mu^y}$, $\boldsymbol{\sigma}^{yx}$ and $\mathbf{D}^{(k)}$ must add up.
\begin{proof}
\begin{equation*}
	\begin{split}
		\left(  
	\begin{bmatrix}
		m \times n \\
		\dots \\
		m \times n
	\end{bmatrix}
	\begin{bmatrix}
		n \times m & \vdots & n\times m
	\end{bmatrix} \right)
	\begin{bmatrix}
		m \times 1 \\
		\dots \\
		m \times 1 
	\end{bmatrix}
	= & 
	\begin{bmatrix}
		m \times n \\
		\dots \\
		m \times n
	\end{bmatrix} \mathbf{y^{(k)}}\\ % end line 1
	\biggl( \begin{bmatrix}
		2m \times n 
	\end{bmatrix}
	\begin{bmatrix}
		n \times 2m
	\end{bmatrix} \biggr) 
	\begin{bmatrix}
		2m \times 1
	\end{bmatrix}
	= & 
	\begin{bmatrix}
		2m \times n 
	\end{bmatrix} \mathbf{y^{(k)}} \\ % end line 2
	\biggl( \begin{bmatrix}
		2m \times  2m
	\end{bmatrix} \biggr) 
	\begin{bmatrix}
		2m \times 1 
	\end{bmatrix}
	= &
	\begin{bmatrix}
		2m \times n 
	\end{bmatrix} \mathbf{y^{(k)}} \\ % end line 3
	\begin{bmatrix}
		2m \times 1
	\end{bmatrix}
	= &
	\biggl(\begin{bmatrix}
		2m \times  2m
	\end{bmatrix} \biggr)^{-1} 
	\begin{bmatrix}
		2m \times n 
	\end{bmatrix} \mathbf{y^{(k)}} \\ % end line 4
	\begin{bmatrix}
		2m \times p
	\end{bmatrix}
	= &
	\begin{bmatrix}
		2m \times n 
	\end{bmatrix} \mathbf{y^{(k)}} \\ % end line 5
	& \\
	\Downarrow & \; Dim(\mathbf{y^{(k)}}) = n \times 1 \\ % end line 6
	& \\
	\begin{bmatrix}
		2m \times 1
	\end{bmatrix}
	= &
	\begin{bmatrix}
		2m \times 1 
	\end{bmatrix} \\ % end line 7
	\end{split}
\end{equation*}
where $m$ is the number of classes and $n$ is the number of vectors.
\end{proof} 
The number of calculations has been reduced by a factor of $p/4$ \cite{stylianou98}.

The variables must be derived sequentially as depicted in Figure~\ref{fig:training_cf}  
\begin{figure}[htbp]
  \centering
   \begin{tabular}[h]{c}
	\xymatrix{ 
\textbf{X}_{LSF}\ar[r] &*+<5mm>[F-,]{\txt{GMM}}\ar[r] &*+<5mm>[F-,]{\mathbf{P},\mathbf{D}} \ar[r] &*+<5mm>[F-,]{\mathbf{\Sigma}^{yx},\boldsymbol{\mu}^y} \\ % end line 1
\textbf{Y}_{LSF}\ar`[rrru][urrr]&&&
	}
  \end{tabular}
  \caption{Training Procedure of the Conversion Function}
  \label{fig:training_cf}
\end{figure}
% section Conversion Function (end)
% chapter Method (end)