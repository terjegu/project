\chapter{Theory} % (fold)
\label{cha:theory}

\begin{figure}[htbp]
  \centering
   \begin{tabular}[h]{c}
\xymatrix{ 
x[n] \ar[r] &*+<5mm>[F]{\text{Buffer}} \ar[d]\ar[r] &*+<5mm>[F]{1-A(z)}\ar[rr]^{e[n]}& &*+<5mm>[F]{\frac{1}{1-\tilde{A}(z)}} \ar[r] & y(x)\\
  &*+<5mm>[F]{\text{LP}} \ar[r]^{A(z)}& \ar[u]\ar[r]&*+<5mm>[F]{\text{VC}}\ar[r]^{\tilde{A}(z)}& \ar[u]&}
  \end{tabular}
  \caption{Voice Conversion}
  \label{fig:ADPCM}
\end{figure}
Buffer amount, LP type, VC function, re-estimate signal.

\section{Gaussian Mixture Model} % (fold)
\label{sec:gaussian_mixture_model}
The GMM is a classical parametric model used in many pattern recognition techniques \cite{stylianou98}. The GMM assumes that the probability distribution of the observed parameters takes the following parametric form
\begin{equation}
	\label{eq:gmm}
	p(\mathbf{x}) = \sum_{i=1}^{m} \alpha_i N(\mathbf{x}; \boldsymbol{\mu}_i, \mathbf{\Sigma}_i)
\end{equation}
where $m$ is the number of mixture models and $N(\mathbf{x}; \boldsymbol{\mu}_i, \mathbf{\Sigma}_i)$ denotes the p-dimensional normal distribution with the mean vector $\boldsymbol{\mu}_i$ and covariance matrix $\mathbf{\Sigma}_i$ defined by
\begin{equation}
	N(\mathbf{x}; \boldsymbol{\mu}, \mathbf{\Sigma}) = \frac{1}{\sqrt{(2\pi)^p\abs{\mathbf{\Sigma}}}} \exp\left[ -\frac{1}{2} (\mathbf{x} -\boldsymbol{\mu})^T \mathbf{\Sigma}^{-1} (\mathbf{x} -\boldsymbol{\mu})\right]
\end{equation}
and the $\alpha_i$ in \eqref{eq:gmm} is a normalised positive scalar; $\sum_{i=1}^{M}\alpha_i = 1$ and $\alpha_i \geq 0$. The input vectors, $\mathbf{x_i}$, are assumed to be independent.

In the GMM, each component is described by its center, $\boldsymbol{\mu}_i$, and the spreading around the center of the component, $\mathbf{\Sigma_i}$. The frequency of each component in the observation is represented by mixture weights, $\alpha_i$ \cite{stylianou98}. The conditional probability that a given observation vector $\mathbf{x}$ belongs to the component $C_i$ of the GMM is given by Bayes' rule \cite{statistikk} as
\begin{equation}
	\label{eq:bayes}
	P(C_i\vert \mathbf{x}) = \frac{\alpha_i N(\mathbf{x}; \boldsymbol{\mu}_i, \mathbf{\Sigma}_i)}{\sum_{j=1}^{m}\alpha_j N(\mathbf{x}; \boldsymbol{\mu}_j, \mathbf{\Sigma}_j)}.
\end{equation}
% section Gaussian Mixture Model (end)

\section{Expectation Maximisation} % (fold)
\label{sec:expectation_maximisation}
The expectation maximisation, EM \abbrev{EM}{Expectation Maximisation}, algorithm is an iterative algorithm for unsupervised learning in which component information is unavailable or only partly available \cite{taletek}. It estimates the model parameters by maximising the log-likelihood of incomplete data and maximising the expectation of log-likelihood from complete data.

We need to determine the parameter $\mathbf{\Phi} = \{\boldsymbol{\alpha}, \boldsymbol{\mu}, \mathbf{\Sigma}\}$ which maximises $P(Y=y\vert \mathbf{\Phi})$ where $y$ is the observed training data. We assume some parameter vector $\mathbf{\Phi}$ end estimate the probability of some unknown data $x$ occurred in the generation of $y$. We then compute a new $\bar{\mathbf{\Phi}}$ which is the maximum likelihood of $\mathbf{\Phi}$ and set the new $\bar{\mathbf{\Phi}}$ to be $\mathbf{\Phi}$ and repeat the process iteratively until the process converges \cite{taletek}. The process will converge if we choose $\bar{\mathbf{\Phi}}$ such that 
\begin{equation}
	\label{eq:q_criteria}
	Q(\mathbf{\Phi},\bar{\mathbf{\Phi}})\geq Q(\mathbf{\Phi},\mathbf{\Phi})
\end{equation}
where 
\begin{equation}
	\label{eq:q_function}
	Q(\mathbf{\Phi},\bar{\mathbf{\Phi}}) = E[\log P(X,Y=y\vert \bar{\mathbf{\Phi}})].
\end{equation}
% section Estimation Maximasation (end)


\section{Conversion Function} % (fold)
\label{sec:conversion_function}
The available data for the conversion function is two time-aligned vectors, source and target, which are of the same length $n$ and the pre-trained gaussian mixture models. A conversion function suggested by Y. Stylianou \cite{stylianou95} is used in this paper:
\begin{defn}
	\label{defn:conversion_function}
	\begin{equation}
		\label{eq:conversion_function}
		\begin{split}
		\mathcal{F}(\mathbf{x}_t) =& E[\mathbf{y}\vert \mathbf{x}_t]\\
		 =& \sum_{i=1}^{m}P(C_i \vert \mathbf{x}_t)[\boldsymbol{\mu}_i^y + \mathbf{\Sigma}_i^{yx} \mathbf{\Sigma}_i^{xx^{-1}} (\mathbf{x}_t-\boldsymbol{\mu}_i^x)]
		\end{split}
	\end{equation}	
\end{defn}
The conversion function is a weighted sum of minimum mean square error (MMSE\abbrev{MMSE}{Minimum Mean Square Error}) estimate of the target vector $\mathbf{y}_t$ given the observed value of $\mathbf{x}_t$, which are jointly Gaussian \cite{stylianou98}. The MMSE is given by \cite{taletek}
\begin{equation}
	\label{eq:mmse}
	E[\mathbf{y}\vert \mathbf{x}=\mathbf{x}_t] = \boldsymbol{\mu}^y + \mathbf{\Sigma}^{yx} \mathbf{\Sigma}^{xx^{-1}} (\mathbf{x}_t-\boldsymbol{\mu}^x)
\end{equation}
where $E[]$ denotes the expectation, $\boldsymbol{\mu}$ is the mean
\begin{equation}
	\boldsymbol{\mu}^y = E[\mathbf{y}]
\end{equation}
and $\mathbf{\Sigma}$ is the cross-covariance matrix
\begin{equation}
	\mathbf{\Sigma}^{yx} = E[(\mathbf{y}-\boldsymbol{\mu}^y)(\mathbf{x}-\boldsymbol{\mu}^x)^T].
\end{equation}

If the gaussian components of the mixture could be separated, the conversion function $\mathcal{F}()$ is modifying the posterior mean and covariance of each component. Since this is not the case in practice the term is weighted by the conditional probability that the observed data $\mathbf{x}_t$ belongs to the component $C_i$.

The variables $\boldsymbol{\mu}^y$ and $\mathbf{\Sigma}^{yx}$ are unknown and need to be estimated such as the squared conversion error is minimised
\begin{equation}
	\label{eq:conversion_error}
	\epsilon=\sum_{t=1}^{n}\norm{\mathbf{y}_t - \mathcal{F}(\mathbf{x}_t)}^2.
\end{equation}

The optimal values of the parameters can be computed by solving the following set of normal equations \cite{stylianou95}:
\begin{equation}
	\begin{split}
		\mathbf{y} = &\mathbf{P}\boldsymbol{\mu^y} + \mathbf{D}\mathbf{\Sigma}^{yx} \\
		= & \begin{bmatrix}
			\mathbf{P}& \vdots &\mathbf{D}
		\end{bmatrix}
		\begin{bmatrix}
			\boldsymbol{\mu^y} \\
			\dots \\
			\mathbf{\Sigma}^{yx}
		\end{bmatrix}
	\end{split}
\end{equation}
\begin{equation}
	\label{eq:param_computed}
	\begin{split}
		\left( 
		\begin{bmatrix}
			\mathbf{P}^T \\
			\dots \\
			\mathbf{D}^T
		\end{bmatrix}
		\begin{bmatrix}
			\mathbf{P} & \vdots & \mathbf{D}
		\end{bmatrix}
		 \right)
		\begin{bmatrix}
			\boldsymbol{\mu}^y \\
			\dots \\
			\mathbf{\Sigma}^{yx}
		\end{bmatrix}
		= &
		\begin{bmatrix}
			\mathbf{P}^T \\
			\dots \\
			\mathbf{D}^T
		\end{bmatrix}
		\mathbf{y} \\ % end line 1
		\begin{bmatrix}
			\boldsymbol{\mu}^y \\
			\dots \\
			\mathbf{\Sigma}^{yx}
		\end{bmatrix}
		= &
		\left( 
		\begin{bmatrix}
			\mathbf{P}^T \\
			\dots \\
			\mathbf{D}^T
		\end{bmatrix}
		\begin{bmatrix}
			\mathbf{P} & \vdots & \mathbf{D}
		\end{bmatrix}
		 \right)^{-1}
		\begin{bmatrix}
			\mathbf{P}^T \\
			\dots \\
			\mathbf{D}^T
		\end{bmatrix}
		\mathbf{y} \\ % end line 2
	\end{split}
\end{equation}
where $\mathbf{y}$ is the target spectral vectors and $\mathbf{P}$ is a $n \times m$ matrix with the conditional probabilities
\begin{equation}
	\label{eq:P_matrix}
	\mathbf{P} = \begin{bmatrix}
		P(C_1\vert \mathbf{x}_1) & \dots & P(C_m\vert \mathbf{x}_1) \\
		P(C_1\vert \mathbf{x}_2) & \dots & P(C_m\vert \mathbf{x}_2) \\
		\vdots & & \vdots \\
		P(C_1\vert \mathbf{x}_n) & \dots & P(C_m\vert \mathbf{x}_n) \\
	\end{bmatrix}.
\end{equation}
$\mathbf{D}$ is a $n \times pm$ matrix defined as
\begin{equation}
	\label{eq:D_matrix}
	\mathbf{D} = \begin{bmatrix}
		P(C_1\vert \mathbf{x}_1)(\mathbf{x}_1 - \boldsymbol{\mu}_1^x)^T\mathbf{\Sigma}_1^{xx^{-1}} & \dots & P(C_m\vert \mathbf{x}_1)(\mathbf{x}_1 - \boldsymbol{\mu}_m^x)^T\mathbf{\Sigma}_m^{xx^{-1}} \\
		P(C_1\vert \mathbf{x}_2)(\mathbf{x}_2 - \boldsymbol{\mu}_1^x)^T\mathbf{\Sigma}_1^{xx^{-1}} & \dots & P(C_m\vert \mathbf{x}_2)(\mathbf{x}_2 - \boldsymbol{\mu}_m^x)^T\mathbf{\Sigma}_m^{xx^{-1}} \\
		\vdots & & \vdots \\
		P(C_1\vert \mathbf{x}_n)(\mathbf{x}_n - \boldsymbol{\mu}_1^x)^T\mathbf{\Sigma}_1^{xx^{-1}} & \dots & P(C_m\vert \mathbf{x}_n)(\mathbf{x}_n - \boldsymbol{\mu}_m^x)^T\mathbf{\Sigma}_m^{xx^{-1}} \\
	\end{bmatrix}.
\end{equation}
The two unknown matrices $\boldsymbol{\mu}^x$ and $\mathbf{\Sigma}^{xx}$ will have dimensions $m\times p$ and $m \times (p\times p)$ 
\begin{equation}
	\label{eq:v_matrix}
	\boldsymbol{\mu}^x = 
	\begin{bmatrix}
		\boldsymbol{\mu}^x_1 \vdots \boldsymbol{\mu}^x_2 \vdots \dots \vdots \boldsymbol{\mu}^x_m
	\end{bmatrix}^T
\end{equation}
\begin{equation}
	\label{gamma_matrix}
	\mathbf{\Sigma}^{xx} = 
	\begin{bmatrix}
		\mathbf{\Sigma}_1^{xx} \vdots \mathbf{\Sigma}_2^{xx} \vdots \dots \vdots \mathbf{\Sigma}_m^{xx}
	\end{bmatrix}^T.
\end{equation}
% section Transfer function (end)
% chapter Theory (end)

\section{Signal Representation} % (fold)
\label{sec:signal_representation}
LPC, RC, MFCC.
% section Signal Representation (end)