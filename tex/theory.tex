\chapter{Theory} % (fold)
\label{cha:theory}
\section{Gaussian Mixture Model} % (fold)
\label{sec:gaussian_mixture_model}
The GMM is a classical parametric model used in many pattern recognition techniques \cite{stylianou98}. The GMM assumes that the probability distribution of the observed parameters takes the following parametric form
\begin{equation}
	\label{eq:gmm}
	p(\mathbf{x}) = \sum_{i=1}^{m} \alpha_i N(\mathbf{x}; \boldsymbol{\mu}_i, \mathbf{\Sigma}_i)
\end{equation}
where $N(\mathbf{x}; \mathbf{\mu}_i, \mathbf{\Sigma}_i)$ denotes the p-dimensional normal distribution with the mean vector $\boldsymbol{\mu}$ and covariance matrix $\mathbf{\Sigma}$ defined by
\begin{equation}
	N(\mathbf{x}; \boldsymbol{\mu}_i, \mathbf{\Sigma}_i) = \frac{1}{\sqrt{(2\pi)^p\abs{\mathbf{\Sigma}}}} \exp\left[ -\frac{1}{2} (\mathbf{x} -\boldsymbol{\mu})^T \mathbf{\Sigma}^{-1} (\mathbf{x} -\boldsymbol{\mu})\right]
\end{equation}
and the $\alpha_i$ in \eqref{eq:gmm} is a normalised positive scalar; $\sum_{i=1}^{M}\alpha_i = 1$ and $\alpha_i \geq 0$. The $\mathbf{x_i}$ vectors are assumed to be independent.

In the GMM, each class is described by its center, $\boldsymbol{\mu}_i$, and the spreading around the center of the class, $\mathbf{\Sigma_i}$. The frequency of each class in the observation is represented by mixture weights, $\alpha_i$ \cite{stylianou98}. The conditional probability that a given observation vector $\mathbf{x}$ belongs to the class $C_i$ of the GMM is given by Bayes' rule \cite{statistikk} as
\begin{equation}
	\label{eq:bayes}
	P(C_i\vert \mathbf{x}) = \frac{\alpha_i N(\mathbf{x}; \boldsymbol{\mu}_i, \mathbf{\Sigma}_i)}{\sum_{j=1}^{m}\alpha_j N(\mathbf{x}; \boldsymbol{\mu}_j, \mathbf{\Sigma}_j)}.
\end{equation}

TODO: BESKRIV PARAMETER ESTIMERING (EM)
% section Gaussian Mixture Model (end)

\section{Harmonic + Noise Model} % (fold)
\label{sec:harmonic_noise_model}
Harmonica + noise model, HNM\abbrev{HNM}{Harmonic + Noise Model}, is a model for high-quality modifications of speech signals.
% section Harmonic + Noise Model (end)

\section{Time-scale modification} % (fold)
\label{sec:time_scale_modification}

% section Time-scale modification (end)

\section{Pitch-scale modification} % (fold)
\label{sec:pitch_scale_modification}

% section Pitch-scale modification (end)

\section{Spectral modification} % (fold)
\label{sec:spectral_modification}

\subsection{Rule-based} % (fold)
\label{sub:rule_based}

% subsection Rule-based (end)

\subsection{Statistical} % (fold)
\label{sub:statistical}

% subsection Statistical (end)
% section Spectral modification (end)
% chapter Theory (end)