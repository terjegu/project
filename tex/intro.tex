\chapter{Introduction} % (fold)
\label{cha:introduction}
One of the earliest implementations of speech conversion utilised vector quantisation\cite{abe88} with discrete source and target classes. In contrast the GMM method \cite{stylianou98} which uses continuous probability density mapping.

\section{Definition} % (fold)
\label{sec:definition}

% section Definition (end)

\section{Motivation} % (fold)
\label{sec:motivation}

% section Motivation (end)

\section{Modification methods} % (fold)
\label{sec:synthesis_methods}
Speech modification can be classified into three categories\cite{nguyen09}.
\begin{description}
	\item[Time-scale modification: ] The process of altering the duration of the speech signal without modifying the pitch and other qualities.
	\item[Pitch-scale modification: ] The goal of pitch modification is to change the fundamental frequency while maintaining the speed and spectral properties of the speech signal.
	\item[Spectral modification: ] Spectral modification alters the spectral attributes without modifying the duration and spectral properties.
\end{description}
% section Synthesis methods (end)

% chapter Introduction (end)