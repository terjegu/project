\chapter{Introduction} % (fold)
\label{cha:introduction}
\abbrev{$a$}{a variable} \abbrev{$\mathbf{a}$}{a vector} \abbrev{$\mathbf{A}$}{a matrix}

\section{Definition} % (fold)
\label{sec:definition}
Voice transformation or conversion refers to the various modifications one may apply to the sound produced by a person, speaking or singing \cite{stylianou08}. More specifically, in this papers voice transformation refers to the process of modifying the speech signal from a speaker so that it sounds like it has been pronounced by a specific target speaker. 
% section Definition (end)

\section{Motivation} % (fold)
\label{sec:motivation}
In text-to-speech (TTS\abbrev{TTS}{test-to-speech}) systems the users may want to have a choice between several voices to which they find most soothing to listen to. To create a new voice for a TTS system from scratch requires a lot of work and money, which may not be in the system providers best interest. If it were possible to only create one synthetic voice and then transform that voice to other voices with only limited training data, say 10 minutes, the system provider could produce any number of voices with ease. 

Voice transformation could also be used to disguise your voice if you wish to keep your identification secret during a conversation, like cryptography for your voice. 
% section Motivation (end)

\section{Modification Methods} % (fold)
\label{sec:synthesis_methods}
When working on a speech signal it is often convenient to refer to the source-filter model \cite{taletek}. This model says that speech is produced by passing a glottal excitation (source) through a time-varying linear filter that models the resonant characteristics of the vocal tract (filter). The signal can be separated into source and filter by doing a linear prediction (LP) analysis and subtract the prediction signal from the real signal. By thing doing this you will end up with the filter described as LP coefficients and the source as the excitation from the subtracted signal.

\subsection{Source Modification} % (fold)
\label{sub:source_modification}
The source, which can be modelled as white noise for unvoiced speech and a pulse train for voice speech, can modify the pace, pitch and loudness of a voice. 

Time- and pitch-scale modifications can be done with the pitch-synchronous overlap-add (PSOLA\abbrev{PSOLA}{pitch-synchronous overlap-add}) procedure. PSOLA splits the signals into windowed parts and re-synthesises the parts with the overlap-add procedure by duplicating or deleting frames to alter the pace of the voice and adjusting the spacing of windows to alter the pitch.
% subsection Source Modification (end)

\subsection{Filter Modification} % (fold)
\label{sub:filter_modification}
One of the earliest implementations of speech conversion utilised vector quantisation (VQ\abbrev{VQ}{vector quantisation}) for mapping spectral properties with discrete source and target classes \cite{abe88}. By training codebooks to represent which filter characteristics match of the source and target speaker, the transformation is simply to loop up in the codebook which filter parameters to swap. In contrast to the Gaussian mixture model (GMM\abbrev{GMM}{Gaussian mixture model}) method which uses continuous probability density mapping \cite{stylianou98}. This method maps speaker specific features from the source speaker to the target speaker according to a transformation function. Another mapping approach is the use of Artificial Neural Networks (ANN\abbrev{ANN}{artificial neural networks}) \cite{desai09} which tries to mimic the computational mechanism of the human brain \cite{young75}.

The filter can modify the magnitude spectrum of the frequency response of the vocal tract \cite{nguyen09}. The filter is often modelled as a linear predictive (LP\abbrev{LP}{linear predictive}) filter with a small set of filter coefficients. These coefficients can be altered to change the formant frequencies and so on. The magnitude spectrum carries information of speaker individuality \cite{stylianou09}. If we alter the spectrum to match a certain target speaker it is called \emph{Filter Mapping}. The phase is not altered in this approach which is unfortunate for the quality of the transformation. The spectral parameters of the source speaker is modelled as Gaussian mixture models.
% subsection Filter Modification (end)

\subsection{Voice Transformation} % (fold)
\label{sub:voice_transformation}
Voice transformation is usually done by a combination of source and filter modifications. However, this paper will not cover source modifications.
% subsection Voice Transformation (end)
% section Synthesis methods (end)

% chapter Introduction (end)