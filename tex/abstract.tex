\begin{abstract}
	Voice transformation or conversion refers to the various modifications one may apply to the sound produced by a person, speaking or singing. More specifically, voice transformation can be used to modifying the speech signal from a speaker so that it sounds like it has been pronounced by a specific target speaker. This can be used in various speech synthesis applications to create several voices from one voice. The old fashion way of doing this requires several hours of speech recordings and even more post-processing. With voice transformation it is sufficient with only minutes of speech and the post-processing is automatic.
	
	This paper explains the implementation of a voice transformation procedure based on Gaussian mixture models suggested by Yannis Stylianou in 1995 \cite{stylianou95}. The speech is modelled as a source-filter model where both parts are characterised by the speaker. The filter is a linear FIR filter with parameters which can be altered to modify the characteristics of the voice to match a certain target speaker.
\end{abstract}