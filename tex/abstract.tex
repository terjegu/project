\begin{abstract}
	Voice transformation, or conversion, refers to the various modifications that can be applied to the sound produced by a person; speaking or singing. More specifically, voice transformation can be used to modify the speech signal from a person so that it sounds like a certain other person has produced it. Voice transformation can be used in various speech synthesis applications, \eg to create new synthetic voices from a pre-manufactured voice. The old-fashioned way of doing this requires several hours of speech recordings and huge amount of post-processing. With voice transformation it is sufficient with only minutes of speech, and post-processing is automatic.
	
	This paper explains the implementation of a voice transformation procedure based on Gaussian mixture models, suggested by Yannis Stylianou in 1995 \cite{stylianou95}. The speech signal is modelled as a source-filter model \cite{taletek} where both parts are unique for all voices. The filter is a linear IIR\abbrev{IIR}{infinite impulse response} filter with parameters that can be altered to modify the frequency spectrum of the voice, to match a certain target speaker. The source speaker is modelled with Gaussian mixture models, trained with six minutes of speech. A transformation function is trained with parallel time-aligned training data from the source and target speaker, in addition to the mixture models.
	
	Various complexity versions of the filter transformation were tested, and the results were promising. The source signal from the source-filter separation was not modified, which is necessary to achieve a good synthesis of the transformed filter. To achieve an acceptable quality for a real world application, the number of mixture models and amount of training data need to be increased, compared to setup used in this paper.
\end{abstract}